%! TeX root: ../notes.tex
\section{Section One}
\label{sets}

Introductory\footnote{Footnote.} things \cite{ireland1990classical}.

\subsection{Definitions and More}

Definitions are blue.
\begin{definition}[Definition Name]
    Define a \vocab{term} here.
\end{definition}

Claims, lemmas, propositions, theorems, corollaries all may come with proofs.

\begin{claim}[Claim Name]
    A claim.
\end{claim}

\begin{lemma}[Lemma Name]
    A lemma.
\end{lemma}

\begin{proposition}[Proposition Name]
    A proposition.
\end{proposition}

\begin{theorem}[Theorem Name]
    A theorem.
\end{theorem}

\begin{proof}
    A proof.
\end{proof}

\begin{corollary}[Corollary Name]
    A corollary.
\end{corollary}

\begin{example}[Example Name]
    Examples
\end{example}

Conjectures, questions, and remarks are food for thought.

\begin{conjecture}[Conjecture Name]
    A conjecture.
\end{conjecture}

\begin{ques}
    \label{ques:set}
    A question.
\end{ques}

\begin{remark}
    Links look like \href{https://en.wikipedia.org/wiki/cantor%27s_diagonal_argument}{this}.
\end{remark}

You can reference things like this: question \ref{ques:set}.
